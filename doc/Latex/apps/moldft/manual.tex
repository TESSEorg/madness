\documentclass[letterpaper]{book}
\usepackage{amssymb,amsmath,latexsym,hyperref,graphics,color}

\setlength{\parindent}{0pt}
\setlength{\parskip}{2ex plus 0.5ex minus 0.2ex}

\begin{document}

% Title Page
\title{MADNESS Molecular electronic structure calculations}
\date{Last Modification: 7/7/2016}
\maketitle

% Copyright Page
\pagestyle{empty}
\null\vfill
\noindent
This file is part of MADNESS.


Copyright (C) 2007, 2010 Oak Ridge National Laboratory

This program is free software; you can redistribute it and/or modify it under the terms of the GNU General Public
License as published by the Free Software Foundation; either version 2 of the License, or(at your option) any later
version.

This program is distributed in the hope that it will be useful, but WITHOUT ANY WARRANTY; without even the implied
warranty of MERCHANTABILITY or FITNESS FOR A PARTICULAR PURPOSE. See the GNU General Public License for more details.

You should have received a copy of the GNU General Public License along with this program; if not, write to the Free
Software Foundation, Inc., 59 Temple Place, Suite 330, Boston, MA 02111-1307 USA

For more information please contact:
\begin{quote}							
Robert J. Harrison 				\\
Oak Ridge National Laboratory 	\\
One Bethel Valley Road 			\\
P.O. Box 2008, MS-6367			\\
Oak Ridge, TN 37831				\\
								\\
email: harrisonrj@ornl.gov 		\\
tel: 865-241-3937				\\
fax: 865-572-0680	
\end{quote}		
\newpage


% Table of Contents Pages
\clearpage
\setcounter{page}{1}
\pagenumbering{roman}

\setcounter{tocdepth}{10}
\renewcommand\contentsname{Table of Contents}
\tableofcontents


\clearpage
\setcounter{page}{1}
\pagenumbering{arabic}

\chapter{Overview}

\section{Capabilities}



\section{Current status}

c

d

\section{Input structure}



e

f

\section{Reproducibility}

g

\chapter{Geometry specification}





\chapter{Ground state molecular DFT and HF}

{\tt nopen value} --- E.g., {\tt nopen 3} --- The number of unpaired spin orbitals, $n_\alpha - n_\beta$ (default {\tt 0})

{\tt unrestricted} --- Selects a spin-unrestricted calculation (default is spin restricted)

{\tt xc value} --- E.g., {\tt xc HF} --- Selects the exchange correlation potential (default is {\tt LDA}). See XC section for more details.

{\tt aobasis value} --- E.g., {\tt aobasis sto-3g} --- Sets the atomic orbital basis used for the initial guess.  Options are {\tt sto-3g} (down to Iodine)  or {\tt 6-31g} (down to Zinc, default).  

{\tt charge value} --- E.g., {\tt charge -1.0} --- Total charge (default {\tt 0}) on the molecule. Atomic units.

{\tt nvalpha value} --- E.g., {\tt nvalpha 2} --- The number of alpha spin virtual orbitals to solve for (default {\tt 0}) --- is this working now?

{\tt nvbeta value} --- E.g., {\tt nvbeta 2} --- The number of beta spin virtual orbitals to solve for (default {\tt 0}) --- is this working now?

{\tt no\_orient} --- Do not reorient/translate the molecule to orientation/center.

{\tt core\_type value} --- E.g., {\tt What is available?} Selects the pseudopotential to be used on all atoms (can also do mixed all-electron/pseudopotential calculation).  Not heavily tested and unoptimized. (default is all electron).

{\tt psp\_calc} --- Perform pseusopotential calculation on all atoms.  Not heavily tested and unoptimized. (default is all-electron)

{\tt L value} --- E.g., {\tt L 50} --- Sets the computational box size to $[-L,L]^3$ atomic units (mostly for testing).  Default is to find cube that contains all nuclei and to add 50 atomic units.

\section{XC --- DFT exchange correlation}

Without LIBXC, the code just provides either Hartree-Fock ({\tt xc HF}) or local (spin) density approximation ({\tt xc LDA}, the default).

With LIBXC, in addition to HF and LDA (default) there are wide varity of GGA and hybrid functionals available --- the ones that have been tested to some extent include
\begin{itemize}
\item PBE --- \verb+xc GGA_X_PBE 1. GGA_C_PBE 1.+
\item PBE0 --- \verb+xc GGA_X_PBE .75 GGA_C_PBE 1. HF_X .25+
\item PW91 --- \verb+???+
\item B3LYP --- \verb+????+
  
\end{itemize}

We have not yet implemented the near linear-scaling algorithm for HF exchange, which as a consequence is fairly slow.

\section{Restarting}

At completion of an HF or DFT calculation, the molecular orbitals are
saved in the files \verb+restartdata.*+ (with one file per I/O server
process).  The projection of the orbtials onto the {\tt sto-3G} AO basis
set is saved into the file {\tt restartaodata}.

{\tt restart} --- Restart from numerical orbitals from a previous calculation (default is no)

{\tt restartao} --- Restart from projection of orbitals onto AO basis
set from a previous calculation (default is no unless doing geometry
optimization).  If a restart file is not found, or the file contains
incompatible data then the default atomic guess is used.

{\tt save value} --- E.g., {\tt save false} --- Boolean flag to save (or not) orbitals at completion (default is true).

\section{Controlling convergence and accuracy}

The default convergence test is on both the 2-norm of change in density per atom (separately for each spin) between iterations and the residual error in each wave function.

\begin{verbatim}
  converged = (da < dconv * molecule.natom()) && 
              (db < dconv * molecule.natom()) &&
              (conv_only_dens || (max_residual < 5.0 * dconv))
\end{verbatim}

{\tt dconv value} --- E.g., {\tt dconv 1e-5} --- SCF convergence criterion (default 1e-4 atomic units).  Suggest decreasing this to {\tt 1e-5} for geometry optimization or property calculations.

{\tt canon} --- Solves for canonical orbitals or eigenfunctions (default is localized orbitals except for atoms and diatomics).

{\tt pm} --- Selects use of the Pipek-Mezy localized orbitals (default).

{\tt boys} --- Selects use of the Boys localized orbitals.

{\tt maxrotn value} --- E.g., {\tt maxrotn 0.1} --- Used to restrict maximum rotation of orbitals (default {\tt 0.25})

{\tt maxiter value} --- E.g., {\tt maxiter 20} --- The maximum number of iterations (default is {\tt 20})

{\tt maxsub value} --- E.g., {\tt maxsub 5} --- The size of the iterative subspace (default is {\tt 5}).  Sometimes it helps to make this larger.

{\tt protocol valuelist} --- E.g., {\tt protocol 1e-4 1e-6 1e-8} --- Sets the solution protocol.  The default is `{\tt 1e-4 1e-6} which means solve first using a truncation threshold of {\tt 1e-4} (using $k=6$) and with a threshold of {\tt 1e-6} (using $k=8$).

{\tt orbitalshift --- E.g., {\tt orbitalshift 0.1} --- Shifts the occupied orbitals down in energy by the given amount (default {\tt 0}).  Is this working?

{\tt k value} --- E.g., {\tt k 8} --- Sets the wavelet order to a fixed value (mostly only used for testing)

{\tt convonlydens} --- Just test on the change in the density for convergence.


\section{Geometry optimization}

By default geometry optimization is performed using the BFGS Hessian update algorithm.  The convergence test is on all of the 2-norm of the gradient, the change in the energy between iterations, and the maximum change in Cartesian coordinates (all in atomic units).  The 

For geometry optimization it is recommended to select {\tt dconv 1e-5} to obtain more accurate gradients.

{\tt gopt} --- Requests optimization of the geometry

{\tt gtol value} --- E.g., {\tt gtol 1e-4} --- Sets the convergence threshold for the 2-norm of the gradient (default {\tt 1e-3}).

{\tt gtest value} --- E.g., {\tt gtest 1e-4} --- Sets the convergence threshold for the maximum change in Cartesian coordinates (default {\tt 1e-3} atomic units).

{\tt gval value} --- E.g., {\tt gval 1e-6} --- Sets the available precision in the energy (default is {\tt 1e-5} atomic units).

{\tt gprec value} --- E.g., {\tt gtest 1e-6} --- Sets the available precision in the gradient (default is {\tt 1e-5} atomic units).

{\tt gmaxiter value} --- E.g., {\tt gmaxiter 100} --- Sets the maximum number of geometry optimization iterations (default is 20).

{\tt algopt value} --- E.g., {\tt algopt SR1} --- Selects the quasi-Newton update method (default is ).  Options are {\tt BFGS} (default) or {\tt SR1} (not heavily tested).  Case sensitive.

\section{Properties}

{\tt derivatives} --- Compute the derivates (default is false).

{\tt dipole} --- Compute the molecular dipole moment (default is false --- why?).

{\tt response} --- TBD

{\tt response\_freq} ---  TBD

{\tt response\_axis} --- TBD

{\tt rconv} --- TBD

{\tt efield} --- TBD

{\tt efield\_axis x y z} --- TBD

{\tt print\_dipole\_matels} --- TBD

\section{Plotting}

Plots are generated to OpenDX files.  In the {\tt moldft} source directory are two useful files

\begin{itemize}
\item {\tt vizit.net} --- An OpenDX visual program that displays a
  molecule (from file {\tt molecule.dx}) along with positive+negative
  isosurfaces (with adjustable value)for a scalar field read from a
  file.

\item {\tt moldx.py} --- A Python program you can run with your {\em
  output} file as standard input to produce a {\tt molecule.dx} file.
  It is important to use your output file since {\tt moldft} will (by
  default) translate and rotate the molecular coordinates.
  
\end{itemize}

{\tt plotmos lo hi} --- E.g., {\tt plotmos 10 12} --- Plots the molecular orbitals in the given inclusive range (default is none).  Orbitals are numbered from zero.  Seems like this needs extending to accomodate unrestricted calculations.

{\tt plotdens} --- Plots the total electronic charge density and, if spin unrestricted, the spin density (default is off).

{\tt plotcoul} --- Plots the total (electronic + nuclear) electrostatic potential (default is off).

{\tt plotnpt value } --- E.g., {\tt plotnpt 501} --- Sets the number of plots used per dimension in the cube of points (default 101).

{\tt plotcell xlo xhi ylo yhi zlo zhi} --- E.g., {\tt plotcell -10 10 -15 15 -10 5} --- Sets the cell (in atomic units) used for plotting (default is the entire simulation cell).

\section{Parallel execution}

{\tt loadbal vnucfac parts} --- E.g., {\tt loadbal 12 2} --- Adjusts data/loadbalance when running in parallel with MPI.  {\tt vnucfac} (default 12) is extra weight associated with nuclear potential and {\tt parts} (default 2) is the number of partitions (or subtrees) per node.  SCF

{\tt nio value} --- E.g., {\tt nio 10} --- The number of MPI processes to use as I/O servers (default is 1)


\end{document}
